\section{Linear system}
\exo{1}{0}
\nosolution

\exo{2}{1}
\begin{solution}
  \begin{itemize}
    \item If $x \in \Ker A$, then $Ax = 0$
      which also means that $RAX = R0 = 0$ so $X \in \Ker(RA)$.
    \item If $y \in \Ima(AR)$, then there is a $x$
      such that $ARx = y$
      which also means that there is a $x'$ such that
      $Ax' = y$ since we can just take $x' = Rx$.
      So $y \in \Ima A$.
  \end{itemize}
\end{solution}

\exo{3}{2}
\begin{solution}
  Let's suppose by contradiction that there are 2 ways
  to do this
  \begin{align}
    \label{eq:2.3.1}
    \mathbf{x} & = \sum_{i=1}^k \alpha_i \mathbf{a}_i\\
    \label{eq:2.3.2}
               & = \sum_{i=1}^k \beta_i \mathbf{a}_i
  \end{align}
  which $\beta_i \neq \alpha_i$ for at least one $i \in \{0, \ldots, k\}$.
  $\eqref{eq:2.3.2}-\eqref{eq:2.3.1}$ gives
  \[
    0 = \sum_{i=1}^k (\beta_i-\alpha_i) \mathbf{a}_i.
  \]
  by hypothesis, there is a $i$ such that $\beta_i-\alpha_i \neq 0$
  which is a contradiction with the linear independence of the $\mathbf{a}_i$.
\end{solution}

\exo{4}{4}
\begin{solution}
  Let's suppose by conradiction that there are 2
  basis $a_1, \ldots, a_n$ and $b_1, \ldots, b_m$ with $n < m$.

  Let's just remember that $a_1, \ldots, a_n$ is spanning
  and $b_1, \ldots, b_m$ are linearly independent.

  We have therefore that
  \[ b_1 = \alpha_1 a_1 + \ldots + \alpha_n a_n\]
  with at least one $\alpha_i \neq 0$ since $b_1 \neq 0$.
  Wlog, let's say that $\alpha_1 \neq 0$.
  Therefore
  \[ a_1 = \frac{1}{\alpha_1}b_1 + \frac{\alpha_2}{\alpha_1} a_2 + \ldots + \frac{\alpha_n}{\alpha_1} a_n\]
  so $(b_1, a_2, \ldots, a_n)$ is spanning.

  We now have
  \[ b_2 = \beta_1 b_1 + \beta_2 a_2 + \ldots + \beta_n a_n\]
  if $\beta_2 = \cdots = \beta_n = 0$, $b_1, b_2$ are not linearly independent.
  Wlog, let's say that $\beta_2 \neq 0$,
  $(b_1,b_2,a_3,\ldots,a_n)$ is therefore spanning.

  Continuing this reasoning, $(b_1,\ldots,b_n)$ is spanning.
  $b_{n+1}$ is therefore a linear combination of $b_1, \ldots, b_n$ which is a contradiction.
\end{solution}

\exo{5}{2}
\begin{solution}
  Since $\mathbb{C}^{m \times n}$ is a vector space and if $X,Y \in \mathbb{C}^{m \times n}$, $\trace(Y^*X) \in \mathbb{C}$ which is a field,
  $\trace(Y^*X)$ could be the scalar product $(X,Y)$.
  Let's check.
  Since
  \begin{align*}
    \trace(X^*X)
    & = \sum_{i = 1}^m \sum_{j = 1}^n \overline{x_{ij}}x_{ij}\\
    & = \sum_{i = 1}^m \sum_{j = 1}^n |x_{ij}|^2,
  \end{align*}
  we clearly have $\trace(X^*X) \geq 0$ for all $X \in \mathbb{C}^{m \times n}$ and
  $\trace(X^*X) \Longleftrightarrow X = 0$.
  We also have
  \begin{align*}
    \trace(Z^*(\alpha X + \beta Y))
    & = \trace(\alpha Z^*X + \beta Z^*Y)\\
    & = \alpha\trace(Z^*X) + \beta\trace(Z^*Y)\\
    \trace(Y^*X)
    & = \trace((X^*Y)^*)\\
    & = \overline{\trace(X^*Y)}.
  \end{align*}
  Since it is a scalar product, we can apply the Schwarz inequality which gives
  \begin{align*}
    |\trace(Y^*X)|
    & \leq \sqrt{\trace(X^*X)} \sqrt{\trace(Y^*Y)}\\
    & = \|X\|_F \|X\|_F.
  \end{align*}
\end{solution}

\exo{6}{0}
\nosolution

\exo{7}{0}
\nosolution

\exo{8}{0}
\nosolution

\exo{9}{0}
\nosolution

\subsection*{2.10}
\nosolution

\subsection*{2.11}
\nosolution

\subsection*{2.12}
\nosolution

\subsection*{2.13}
\nosolution

\subsection*{2.14}
\nosolution

\subsection*{2.15}
\nosolution

\subsection*{2.16}
\nosolution

\subsection*{2.17}
\nosolution

\subsection*{2.18}
\nosolution

\subsection*{2.19}
\nosolution

\subsection*{2.20}
\nosolution
